\section{Arbeitspakete}
Dieser Abschnitt beschreibt die Unterteilung des Projektes in Arbeitspakete und wie die Gruppenmitglieder die zugeteilten Arbeitspakte abarbeitet haben. Es wird auch auf den unvorhergesehenen Arbeitsbereitschaftsmangel der Gruppenmitglieder Herr Gregarek und Herr Dück eingegangen.

\subsection{Einteilung der Arbeitspakete}
Zunächst wurden die Arbeitspakete aus der Anforderungsanalyse abgeleitet. Aus dieser Ableitung sind folgende Arbeitspakete definiert worden:
\begin{enumerate}
	\item Entwicklung des grundlegenden Webseitendesigns für Desktop-Rechner
	\item Anpassung des Designs für mobile Geräte
	\item Entwicklung der Webseitennavigationsleiste
	\item Entwicklung der inhaltlichen Seiten des Webshops:
	\begin{itemize}
		\item Startseite
		\item Produktseite
		\item Rechtliche Seiten:
		\begin{itemize}
			\item[$\diamond$] Impressum
			\item[$\diamond$] Datenschutz
			\item[$\diamond$] AGB
		\end{itemize}
	\end{itemize}
	\item Entwicklung einer Schnittstellendatenbank zum Datenaustausch zwischen Webshop und SAP-System:
	\begin{itemize}
		\item Ansprechpartner der SAP-Gruppe für die Schnittstelle zum Webshop
		\item Entwurf von Tabellen für die Datenhaltung des Webshops 
	\end{itemize}
	\item Registrierungs- und Anmeldungsfenster designen und programmieren
	\begin{itemize}
		\item Möglichkeit den Kunden bieten die Anmeldeinformationen später verändern zu können
	\end{itemize}
	\item Entwickeln einer Suchfunktion:
	\begin{itemize}
		\item Autovervollständigung
		\item Erweiterte Suchfunktion, wo die Suche genauer eingegrenzt werden kann
		\item Auflistung der Suchergebnisse
	\end{itemize}
	\item Warenkorb
	\begin{itemize}
		\item Auflistung der Produkte, die sich im Warenkorb befinden
		\item Formular mit Mengenfeld, um ein Produkt in den Warenkorb zu packen
		\item Möglichkeit zur nachträglichen Änderung der Produkte im Warenkorb:
		\begin{itemize}
			\item[$\diamond$] Änderung der Bestellmenge
			\item[$\diamond$] Entfernen des Produkts aus dem Warenkorb
		\end{itemize}
	\end{itemize}
	\item Bestellvorgang:
	\begin{itemize}
		\item Bestellroutine mit folgenden Schritten:
		\begin{itemize}
			\item[$\diamond$] Wahl der Versandart
			\item[$\diamond$] Nachträglichen Änderung der Bestellmenge der Produkte
			\item[$\diamond$] Wahl Zahlungsart
			\item[$\diamond$] Wahl Lieferadresse
			\item[$\diamond$] Bestellübersichtseite zur Kontrolle vor den Absenden der Bestellung
		\end{itemize}
		\item Einsicht des aktuellen Status der Bestellung
		\item Möglichkeit die Bestellung zu stornieren
	\end{itemize}
	
	\item Erstellung eines Marketing Mix
	\item Bewertungsfunktion von Artikel (mit Kommentarfunktion)
	\item Email-Versand
	\begin{itemize}
		\item Klasse für den Email-Versand entwerfen
		\item Designen der Bestätigungs-Emails
	\end{itemize}					 
\end{enumerate}

Die oben genannten Arbeitspakete hat sich die Gruppe untereinander aufgeteilt. Stefan Schnürer hat Arbeitspakete 1,3 und 4 übernommen. Benedikt Brüntrup übernahm die Arbeitspakete 5, 6 und 7. Michael Dück hatte sich bereit erklärt die Arbeitspakte 2, 11 und 12 zu bearbeiten. Raphael wollte die Arbeitspakete 8, 9 und 10 abarbeiten.


%Nachfolgend soll auf die einzelnen Zwischenschritte des Projektes eingegangen werden.Das Projekt beginnt mit der Projektplanung. Hierbei werden die Arbeitspakete definiert und an den entsprechenden Gruppenmitgliedern verteilt. Pro Arbeitspaket wird eine Bearbeitungsdauer von zwei Wochen festgelegt. Es wird ein Projektzeitplan sowie eine Risikoanalyse erstellt. Im Anschluss daran werden grobe Design Entwürfe für die Homepage erstellt. Es wird sich für den Design Entwurf von Herrn Brüntrup entschieden. Die Homepage wird in einem matten Grün-Ton designt, um den Kunden die Umweltaspekte von Fahrrädern zu verdeutlichen.
%\textbf{Was machen Herr Gregarek und Herr Dück?}


\subsection{Arbeitspaket von Herrn Schnürer}

Dieser Abschnitt beschreibt den Aufgabenteil von Herrn Schnürer und wie dieser den Seitenprototyp, den statischen und dynamischen Inhalt und Aufbau der Seite sowie die CSS-Datei zur Anzeige des Webshops auf Desktop Rechnern entwickelt hat.


\subsubsection{Arbeitspaket 1: Erster Seitenprototyp}

Der erste Prototyp der Seite basiert auf den Standardverfahren einer Webseite. Dabei wird der Inhalt der Seite in HTML geschrieben, während eine CSS Datei für das Seitenlayout der Webseite verantwortlich ist. \textit{Abbildung 14} zeigt einen ersten groben Designentwurf der Startseite, anhand dessen orientiert sich auch der erste Seitenprototyp von Herrn Schnürer:

\begin{figure}[H]
\begin{center}
\includegraphics[width=150mm]{Bilder/Abbildung2-GroberDesignEntwurfDesWebshops.png}
\end{center}
\caption{Erster grober Designentwurf der Startseite}
\end{figure}


\subsubsection{Arbeitspaket 1 und 4: Startseite des Webshops}

Die Startseite wird zunächst mit zwei einfachen Artikeln versehen. Sie sollen eine grobe Vorstellung liefern, wie die fertige Startseite aussehen könnte.

\subsubsection{Arbeitspaket 3: Seitennavigation}

In den darauf folgenden Wochen liegt der Fokus auf die Seitennavigation in der Kopfzeile (nachfolgend Header) der Seite. Die Seitennavigation wird mithilfe von CSS an den Designentwurf angepasst. Es werden die Hauptkategorien \glqq{Fahrräder}\grqq{}, \glqq{Zubehör}\grqq{}, \glqq{Fahrradteile}\grqq{}, \glqq{Fahrradbekleidung}\grqq{}, \glqq{Marken}\grqq{} und eine Rubrik \glqq{HowTo}\grqq{} angezeigt. Letztere verlinkt auf eine Internetquelle, mit dessen Hilfe die Seitennavigation umgesetzt wurde. Diese Kategorie dient den anderen Gruppenmitgliedern als Referenz und wird in späteren Versionen des Webshops wieder entfernt. Fährt der Nutzer mit der Maus über die Kategorie \glqq{Fahrräder}\grqq{} öffnet sich ein Drop-Down-Menü, welches den Nutzer verschiedene Fahrradmodelle auswählen lässt. Die komplette Seitennavigation wurde in CSS3 umgesetzt. CSS3 kann von nahezu jeden aktuellen Browser gelesen werden und bietet zudem den Vorteil, dass der entsprechende Code in der bisherigen CSS-Datei eingefügt werden kann. Da CSS3 unabhängig von Java Script ist, lässt sich die Seite zudem problemlos bedienen, wenn Java Script auf den jeweiligen Browsern deaktiviert ist. Dies begründet die Entscheidung CSS3 für die Seitennavigation zu verwenden.
\\
\textit{Abbildung 15} und \textit{16} zeigen den bisherigen Seitenprototypen. \textit{Quellcode 1} veranschaulicht die Umsetzung des Dropdown-Menüs in CSS:

\begin{figure}[H]
\begin{center}
\includegraphics[width=150mm]{Bilder/Abbildung3-Seitenprototyp.png}
\end{center}
\caption{Seitenprototyp}
\end{figure}


\begin{figure}[H]
\begin{center}
\includegraphics[width=150mm]{Bilder/Abbildung4-SeitenprototypMitDropDownMenue.png}
\end{center}
\caption{Seitenprototyp mit Drop-Down-Menü}
\end{figure}

\newpage
\begin{center}
	\begin{lstinputlisting}[language=CSS, caption={Auszug aus der CSS Datei - Die Seitennavigation}]
		{Quellcode/statischeNavigation.css}
	\end{lstinputlisting}
\end{center}


%Ein paar Wochen später wird sich dazu entschlossen das Programm „Eclipse“ zum Programmieren des Webshops zu nutzen. Das Programm bietet neben einer Autovervollständigung und der Unterstützung mehrerer Programmiersprachen noch zusätzlich den Vorteil, dass es durch Plug-Ins erweiterbar ist. So lässt sich „Eclipse“ durch das Plug-In „GitHub“ erweitern. „GitHub“ ist eine Cloud basierte Lösung zum gemeinsam bearbeiten von Programmcode. Das komplette Projekt wird dabei auf einen Server hochgeladen. Jedes Gruppenmitglied, das einen Acount bei „GitHub“ erstellt hat, kann dem Projekt beitreten, indem es bei den Gruppen- Administrator um Erlaubnis bittet. Dadurch kann ein Dokument von mehreren Gruppenmitgliedern gemeinsam bearbeitet werden. So entstehen mehrere Versionen von einem Dokument, die anschließend wieder zu einem Dokument zusammengefügt werden können.%

\subsubsection{Arbeitspaket 4: Weitere Arbeiten an der Startseite des Webshops}

Im Laufe der nächsten Woche wird die Startseite des Webshops gemäß dem Designentwurfs erstellt (siehe \textit{Abbildung 17}). Auf der Startseite ist das \glqq{Angebot des Tages}\grqq{} zu sehen. Es gibt eine Kurzbeschreibung zu dem Artikel mit den wichtigsten Eigenschafen. Der untere Teil der Startseite zeigt die meist angesehenen Artikel des Nutzers an.

\begin{figure}[H]
\begin{center}

\includegraphics[width=150mm]{Bilder/Abbildung5-StartseiteDesWebshops.png}
\end{center}
\caption{Startseite des Webshops}
\end{figure}


\subsubsection{Arbeitspaket 4: Rechtsschutz}

In den nächsten Tagen wird das Thema Rechtsschutz behandelt. Dem Webshop werden ein Impressum, die AGB sowie Hinweise zum Datenschutz hinzugefügt:

\begin{figure}[H]
\begin{center}
\includegraphics[width=150mm]{Bilder/Abbildung6-ImpressumDesWebshops.png}
\end{center}
\caption{Impressum des Webshops}
\end{figure}

\begin{figure}[H]
\begin{center}
\includegraphics[width=150mm]{Bilder/Abbildung7-AGBDesWebshops.png}
\end{center}
\caption{AGB des Webshops}
\end{figure}

\begin{figure}[H]
\begin{center}
\includegraphics[width=150mm]{Bilder/Abbildung8-DatenschutzDesWebshops.png}
\end{center}
\caption{Datenschutz des Webshops}
\end{figure}

\subsubsection{Arbeitspaket 3: Dynamische Seitennavigation}

In den nächsten Wochen wird der statische Inhalt des Webshops durch dynamischen Inhalt ersetzt. Das bedeutet, dass die Daten fortan aus einer Datenbank geladen werden. Diese Datenbank wird mit der Datenbank aus dem SAP- System synchronisiert. Zunächst wird die Navigation des Webshops durch dynamische Inhalte ersetzt. Wird nun eine neue Haupt- oder Unterkategorie hinzugefügt oder entfernt, ändert sich die Navigation entsprechend. Zum besseren Verständnis ein kurzes Beispiel: Angenommen die GBI würde jetzt auch Helmkameras anbieten, so würde in der Datenbank eine neue Kategorie \glqq{Helmkameras}\grqq{} erstellt und zur ihr Kameramodelle von \glqq{GoPro}\grqq{} und \glqq{Rollei}\grqq{}. Die Navigation hätte jetzt die Hauptkategorien \glqq{Fahrräder}\grqq{}, \glqq{Zubehör}\grqq{}, \glqq{Fahrradteile}\grqq{}, \glqq{Fahrradbekleidung}\grqq{} und die Hauptkategorie \glqq{Helmkameras}\grqq{}. Unter letzterer würden die beiden Unterkategorien \glqq{GoPro}\grqq{} und \glqq{Rollei}\grqq{} aufgeführt. \textit{Abbildung 21} zeigt die dynamische Navigation.
\textit{Quellcode 2} zeigt den entsprechenden Quellcode der die Methoden der Klasse Navigation aufruft, um diese dynamisch zu erstellen:

\begin{figure}[H]
\begin{center}
\includegraphics[width=150mm]{Bilder/Abbildung9-DynamischeNaviagtionDesWebshops.png}
\end{center}
\caption{Dynamische Navigation}
\label{Abbildung9-Dynamische Navigation}
\end{figure}

\newpage
\begin{center}
	\begin{lstinputlisting}[language=PHP, caption={Auszug aus der Klasse Navigation - Die dynamische Seitennavigation}]
		{Quellcode/navigation.php.inc}
	\end{lstinputlisting}
\end{center}


\subsubsection{Arbeitspaket 4: Dynamische Artikelauflistung}

Ein paar Tage später erstellt Herr Schnürer eine dynamische Artikelauflistung. In ihr werden alle Fahrräder einer bestimmten Kategorie aufgelistet. Wählt der Nutzer in der Seitennavigation die Hauptkategorie \glqq{Fahrrad}\grqq{} und anschließend die Unterkategorie \glqq{Cityrad}\grqq{} werden ihm alle \glqq{Cityräder}\grqq{} aufgelistet, wählt er die Unterkategorie \glqq{Mountainbike}\grqq{} werden ihm alle \glqq{Mountainbikes}\grqq{} aufgelistet usw. \textit{Abbildung 22} zeigt die dynamische Artikelauflistung aller \glqq{Cityräder}\grqq{} (Anmerkung: Es wird an dieser Stelle nur ein Cityrad aufgelistet, da sich nur ein Produkt mit der Kategorie \glqq{Cityrad}\grqq{} in der Datenbank befindet. Man müsste nur weitere Cityräder in der Datenbank hinterlegen, damit entsprechend mehr Produkte aufgelistet werden):

\begin{figure}[H]
\begin{center}
\includegraphics[width=150mm]{Bilder/Abbildung10-DynamischeArtikelauflistungAllerCitybikes.png}
\end{center}
\caption{Dynamische Artikelauflistung aller Cityräder}
\label{Abbildung10-Dynamische Artikelauflistung aller Cityräder}
\end{figure}


\subsubsection{Arbeitspaket 4: Detaillierte dynamische Artikelbeschreibung}

Eine Woche später wird eine detaillierte und dynamische Beschreibung für die Artikel erstellt. Wählt der Nutzer einen Artikel aus, erscheint eine detaillierte Beschreibung dessen. Der oberer Teil ist dabei an die Artikelbeschreibung der Startseite angelehnt und zunähst noch statisch, der untere Teil beschreibt den Artikel näher und ist bereits dynamisch umgesetzt. So erscheint unter der Überschrift \glqq{Artikelbeschreibung}\grqq{} die jeweilige Beschreibung des zuvor ausgewählten Artikels. \textit{Abbildung 23} zeigt eine solche Beschreibung für das Fahrrad \glqq{Ortler deGoya Damen (2015)}\grqq{}:

\begin{figure}[H]
\begin{center}
\includegraphics[width=150mm]{Bilder/Abbildung11-DynamischeDetailierteArtikelbeschreibung.png}
\end{center}
\caption{Detaillierte dynamische Artikelbeschreibung}
\label{Abbildung11-Detaillierte dynamische Artikelbeschreibung}
\end{figure}

Wie bereits erwähnt ist der obere Teil zunähst nur statisch programmiert worden, weshalb die Überschrift noch nicht zum ausgewählten Artikel passt. Der Text \glqq{Artikelbeschreibung}\grqq{} hingegen wird bereits aus der Datenbank geladen.
\\
In den Sommersemesterferien hat Herr Schnürer auch den oberen Teil dieser Seite dynamisch erstellt. Die Produktbilder und Eigenschaften des Artikels werden nun ebenfalls aus der Datenbank geladen. Zudem wird noch ein kleines Extra eingebaut: Klickt der Nutzer eines der Produktbilder an, wird mithilfe eines Java Scripts eine \glqq{Lightbox}\grqq{} aufgerufen. Diese zeigt die Produktbilder in ihrer normalen Auflösung und ermöglicht das einfache Durchklicken der Produktbilder. \textit{Abbildung 24} zeigt den fertigen oberen Teil der Artikelbeschreibung, während \textit{Abbildung 25} die \glqq{Lightbox}\grqq{} veranschaulicht. \textit{Quellcode 3} zeigt, wie dabei vorgegangen wurde:

\begin{figure}[H]
\begin{center}
\includegraphics[width=150mm]{Bilder/Abbildung12-DynamischeDetailierteArtikelbeschreibungFertig.png}
\end{center}
\caption{Abbildung 12 Fertige dynamische Artikelbeschreibung}
\label{Abbildung12-Fertige dynamische Artikelbeschreibung}
\end{figure}

\begin{figure}[H]
\begin{center}
\includegraphics[width=150mm]{Bilder/Abbildung13-Lightbox.png}
\end{center}
\caption{Abbildung 13 \glqq{Lightbox}\grqq{} zur besseren Darstellung der Produktbilder}
\label{Abbildung13-"Lightbox" zur besseren Darstellung der Produktbilder}
\end{figure}

\newpage
\begin{center}
	\begin{lstinputlisting}[language=PHP, caption={Auszug aus der Seite Artikel Detailansicht}]
		{Quellcode/ArtikelDetailansicht.php.inc}
	\end{lstinputlisting}
\end{center}

%=======================================================
%Startseite: Angebot des Tages noch beschreiben!!!
%=======================================================

\subsubsection{Arbeitspaket 4: Startseite-Angebot des Tages}

Zum Ende des Projektes wurde die Startseite des Webshops fertiggestellt. Sie zeigt das „Angebot des Tages“ in der oberen Hälfte. In der unteren Hälfte werden die meist angesehenen Artikel des Nutzers angezeigt. \textit{Abbildung 26} zeigt noch einmal die fertige Startseite:

\begin{figure}[H]
\begin{center}
\fbox{
\includegraphics[width=150mm]{Bilder/Abbildung14-StartseiteAngebotDesTages.png}
}
\end{center}
\caption{Abbildung 14 Startseite-\glqq{Angebot des Tages}\grqq{}}
\end{figure}



\subsection{Arbeitspakete von Herrn Brüntrup}
In diesem Abschnitt wird beschrieben, wie Herr Brüntrup seine Arbeitspakete gelöst hat. Zudem wird in diesem Abschnitt noch beschrieben welche Arbeitspakete wegen Zeitmangel und fehlender Arbeitsbereitschaft zusätzlich noch übernommen wurden.

\subsubsection{Arbeitspaket 5: Die Schnittstellendatenbank}
Die Datenbank stellt die Schnittstelle zwischen dem SAP-System und dem Webshop da. Aus diesem Grund wurde die Datenbankstruktur in der ersten Woche nach der Arbeitspaketverteilung zusammen mit Mitgliedern der SAP-Gruppe entworfen.\\

\textbf{Gründe für die Schnittstellendatenbank}\\
Die Gründe warum die Kommunikation mit einer Schnittstellendatenbank gelöst wurde, wird in den folgenden Sätzen genannt. Der erste Grund war die Verfügbarkeit. Der Webshop sollte auch erreichbar sein, wenn mal keine Verbindung zum SAP-System besteht. Der SAP-Server steht extern bei einer Hochschule in Magdeburg. Es könnte also schon mal dazu kommen, dass das System nicht erreichbar ist. Ein anderer Grund war die Flexibilität. Die Datenhaltung sollte flexibel erweiterbar sein. Soll die Webseite z.B. durch eine 360$^\circ$-Ansicht des Fahrrads erweitert werden, können einfach die benötigten Felder zur Datenbank hinzugefügt werden. Beim SAP-System ist das nur beschränkt möglich. Der Zugriff auf das SAP-System wird mit vordefinierten Funktionen gelöst. Diese Funktionen werden \glqq Business Application Programming Interface (BAPI)\grqq{} genannt. Da die SAP-Gruppe beim SAP-System keine Berechtigungen hat neue BAPIs zu erstellen, wäre somit die Erweiterbarkeit des Webshops sehr eingeschränkt.\\

\textbf{Aufbau der Schnittstellendatenbank}\\
Nun wird der zu Beginn geplante Aufbau der Schnittstellendatenbank erläutert. Die zu Beginn geplante Datenbank bestand aus den Relationen \glqq Kunde\grqq{}, \glqq Bestellungen\grqq{}, \glqq Produkte\grqq{} und \glqq Produktbilder\grqq{}. Die Relation \glqq Kunde\grqq{} enthält die Login-Daten des Kunden sowie dessen Adresse. Die Relation \glqq Bestellungen\grqq{} enthält alle Angaben zur Bestellung. Angaben sind z.B. die Zahlungs- und Versandart sowie der Status der Bestellung. Die vorletzte genannte Relation beinhaltet alle Angaben zu den Produkten, die beim Webshop verkauft werden sollen. Ihr wurde deswegen auch der Name \glqq Produkte\grqq{} gegeben. Die Relation \glqq Produktbilder\grqq{} enthält Pfadangaben, wo auf den Server die Produktbilder abgelegt sind. Da zu einen Produkt mehrere Produktbilder abgelegt werden können, wurde sich auf eine extra Tabelle für die Produktbilder geeinigt.

Eine Relation, die die Produkte einer Bestellung enthält, wurde zunächst vergessen umzusetzen. Das Vergessen der Relation fiel erst auf, als die SAP-Gruppe mit der Umsetzung des Bestellvorgangs bei der SAP-Schnittstelle beginnen wollte. Zur Lösung des Problems fügte die SAP-Gruppe nun nachträglich eine Relation \glqq Bestellprodukte\grqq{} zur Datenbank hinzu. Diese Relation steht in Beziehung mit den Relationen \glqq Bestellungen\grqq{} und \glqq Produkte\grqq{}.

Als Datenbank-Server wurde sich auf MYSQL geeinigt. Dieser Server lässt sich problemlos auf Linux installieren. Es war wichtig, dass die Datenbank unter Linux läuft, da die Server-Gruppe einen Linux-Server aufsetzen wollte. Zudem bietet MYSQL den Vorteil, dass auf eine MYSQL-Datenbank mit der Programmiersprache PHP gut zugegriffen werden kann. Die SAP-Gruppe wollte die Datenbank mit einer selbst geschriebenen JAVA-Anwendung füllen. Auch von JAVA aus kann gut auf eine MYSQL-Datenbank zugegriffen werden. Ein weiterer Grund für die Nutzung der MYSQL-Datenbank war, dass die Softwaresammlung XAMPP einen MYSQL-Server beinhaltet. Die Softwaresammlung XAMPP wurde von der Webshop-Gruppe als Testserver zum Testen der Webseite auf den Hochschulrechnern verwendet. \\
Die \textit{Abbildung 27} zeigt das \glqq Entity-Relationship-Diagramm (ERD)\grqq{}. Dieses Diagramm ist die erste Version der Datenbank, die zusammen mit der SAP-Gruppe entworfen wurde. 

\begin{figure}[H]
	\begin{center}
		\fbox{
			\includegraphics[width=150mm]{Bilder/Abbildung10-erd_alt.jpg}
		}
	\end{center}
	\caption{Erster Datenbankentwurf}
\end{figure}

Da es ursprünglich bemängelt wurde, dass das Projekt noch nicht den für das Modul benötigten Zeitaufwand aufweisen wurde, wurde beschlossen das Arbeitspaket \glqq Bewertung\grqq{} mit einer Kommentarfunktion zu erweitern. Somit sollte der Kunde auch ein Kommentar für seine Bewertung äußern können. Dafür musste die Datenbank etwas angepasst werden. Vorher war einfach ein Attribut \glqq Bewertung\grqq{} in der Relation \glqq Produkte\grqq{} vorzufinden. Um die Kommentare für die Bewertung speichern zu können, wurde die Datenbank um eine Relation \glqq Kommentare\grqq{} erweitert. Diese Relation beinhaltet ein Datenfeld, wo die Bewertung des Kunden (von 1-5 Sterne) gespeichert wird. Ein weiteres Datenfeld der Relation enthält den Bewertungstext. \\
Als mit der Suchfunktion und der Navigation begonnen wurde, fiel schnell auf, dass benötigte Attribute und Tabellen für die Navigation vergessen wurden. Somit wurde eine neue Relation \glqq Produktkategorie\grqq{} zur Datenbank hinzugefügt. Diese Relation steht mit der Relation \glqq Produkte\grqq{} in Beziehung und beinhaltet alle Produktkategorien, die der Webshop anbieten soll. Unter Produktkategorien wird beim Webshop die Festlegung verstanden, ob das Produkt ein Ersatzteil, ein Fahrrad, Zubehör usw. ist. Eine zweite Relation, die mit der Relation Produkte in Beziehung steht, ist die Relation \glqq Bauart\grqq{}. Dieses Relation wird nur verwendet, wenn das Produkt ein Fahrrad ist. In dieser Relation sind alle beim Webshop angebotenen Bauarten aufgelistet. Unter Bauart wird bei diesen Webshop verstanden, ob es sich um ein Mountainbike, Trekkingbike usw. handelt.

Die \textit{Abbildung 28} zeigt die endgültige Struktur der Schnittstellendatenbank mit allen nachträglich hinzugefügten Relationen. 
\begin{figure}[H]
	\begin{center}
		\fbox{
			\includegraphics[width=150mm]{Bilder/Abbildung11-erd.jpg}
		}
	\end{center}
	\caption{Endgültiger Datenbankentwurf}
\end{figure}

\textbf{Wahl der PHP-API zum Datenbankzugriff}\\
Um von PHP aus auf eine MYSQL-Datenbank zugreifen zu können, gibt es mehrere Möglichkeiten. In dieser Ausarbeitung werden die Zugriffsmöglichkeiten kurz verglichen und dann erläutert, welche davon für das Projekt benutzt wurde.

Die erste Möglichkeit ist die PHP-API \glqq ext/mysql \grqq{}. Diese API ist veraltet und wird nicht mehr weiterentwickelt. Sie unterstützt noch nicht alle MYSQL 5.1+ Funktionalitäten und ist stellenweise mit neuen PHP-Versionen nicht mehr nutzbar. Diese Erweiterung nutzt den prozeduralen Ansatz. Die nächste Möglichkeit ist die PHP-API \glqq PDO\grqq{}. Diese API unterstützt fast alle Funktionalitäten von MYSQL 5.1+. Sie bietet zudem den Vorteil, dass die API auch zum Zugriff auf andere SQL-basierten Datenbanken genutzt werden kann. Diese API hat einen objektorientierten Aufbau.  Die letzte Zugriffsmöglichkeit bietet die API \glqq ext/mysqli\grqq{}. Sie unterstützt alle MYSQL 5.1+ Funktionalitäten und kann objektorientiert und prozedural programmiert werden.

Die Auswahl fiel auf die API \glqq ext/mysqli\grqq{}. Die Gründe hierfür waren, dass die API alle Funktionalitäten der neusten MYSQL-Version unterstützt, regelmäßig auf die neuesten MYSQL-Funktionen angepasst wird und schon Quellcodes von alten Projekten übernommen werden konnte. Die \textit{Abbildung 29} zeigt die Eigenschaften der drei APIs nochmal in einer Tabelle zusammen gefasst.
\begin{figure}[H]
	\begin{center}
		\fbox{
			\includegraphics[width=150mm]{Bilder/Abbildung12-mysql_apis.png}
		}
	\end{center}
	\caption{Überblick der PHP-APIs zum Datenbankzugriff}
\end{figure}

\textbf{Die Konfigurationsdatei}\\
Um auf eine MYSQL-Datenbank zugreifen zu können, werden üblicherweise mehrere Zugangsdaten benötigt. Die Zugangsdaten, die mindestens benötigt werden, sind der Datenbankbenutzername, das Passwort zum Benutzername, der Datenbankname und die IP-Adresse des Servers. Damit diese Zugangsdaten nicht bei jeden Datenbankzugriff im Quelltext angegeben werden müssen, wurden diese Angaben in eine Konfigurationsdatei ausgelagert. Wird die Webseiten später mal auf einen anderen Webserver betrieben, so müssen die Datenbankzugangsdaten nur in dieser Datei geändert werden. Die Konfigurationsdatei ist extra gesichert worden, damit sie nicht direkt im Browser geöffnet werden kann. Der Datei wurde als Dateiendung \glqq *.php.inc\grqq{} angehangen. Mit einer bestimmten Zeile in einer Konfigurationsdatei für den Webserver wurde das Öffnen von \glqq *.inc\grqq{}-Dateien den Browser untersagt. Die Konfigurationsdatei trägt den Namen \glqq .htaccess\grqq{} und befindet sich im Home-Vezeichnis der Webseite. Mehr zur \glqq .htaccess\grqq{}-Datei kann im Abschnitt \glqq Sicherheit\grqq{} gelesen werden.

Der \textit{Quellcode 1} zeigt die Konfigurationsdatei des Webshops. In dieser Datei sind die benötigten Datenbankzugangsdaten und die festgelegte Email-Adresse zu lesen, von der alle Bestätigungs-Emails versendet werden sollen. Zudem sind in der Konfigurationsdatei die Bankdaten definiert, die dem Kunden mitgeteilt werden, wenn er als Zahlungsart \glqq Vorkasse\grqq{} ausgewählt hat.
\newpage
\begin{center}
	\begin{lstinputlisting}[language=PHP, caption={Die Konfigurationsdatei}]
		{Quellcode/config.php.inc}
	\end{lstinputlisting}
\end{center}

\textbf{Die Datenbankzugriffsklasse}\\
Beim Webshop wird nicht direkt über die \glqq MYSQLi\grqq{}-API auf die MYSQL-Datenbank zugegriffen. Es wurde für den Datenbankzugriff extra eine Klasse entwickelt, die als Schnittstelle zwischen den Webshop und der \glqq MYSQLi\grqq{}-API fundiert. Diese Klasse bestand schon von alten Projekten und wurde aus folgendem Grund entwickelt. Sie wurde entwickelt, damit nur an einer Stelle das Projekt verändert werden muss, wenn zu einen späteren Zeitpunkt mal ein anderes API verwendet werden soll. Dieses ist z.B. der Fall, wenn zu eine andere Datenbank verwendet werden soll.

\newpage
\subsubsection{Arbeitspaket 6: Registrierungs- und Anmeldungsfenster designen und programmieren}
Dieser Abschnitt beschreibt, wie beim Webshop die Registrierung und Anmeldung umgesetzt wurde. Zudem wird die Umsetzung einer Funktion des Webshops beschrieben, wo der Kunde seine Profildaten ändern kann.\\

\textbf{Der Registrierungsvorgang}\\
Auf dieser Webseite läuft die Registrierung wie in den folgenden Sätzen beschrieben ab. Zunächst tippt der Neukunde die von der Webseite verlangten persönlichen Informationen in ein Formular ein und bestätigt die Eingabe durch einen Klick auf den Bestätigungsbutton. Wenn beim Webbrowser des Neukunden JavaScript nicht deaktiviert ist, überprüft der Webbrowser mit einem JavaScript die Eingabe bevor sie zum Webserver weitergeleitet wird. Die Eingabe wird nur weitergeleitet, wenn diese vom JavaScript aus als gültig anerkannt wurde. Bei ungültiger Eingabe wird auf der Webseite eine Fehlermeldung ausgegeben. Der JavaScript ist keine endgültige Eingabeüberprüfung, dieser dient nur dafür unnötigen Datenverkehr zu minimieren. JavaScript besitzt den Nachteil, dass dieser beim Webbrowser deaktiviert werden kann. Bei deaktivierten JavaScript wird somit die Eingabe unüberprüft an den Webserver weitergeleitet. Somit muss die Eingabe nochmal beim Webserver auf Gültigkeit überprüft werden.  Der Neukunde wird also erst in die Kunden-Tabelle der Datenbank geschrieben, wenn die serverseitig programmierten Überprüffunktionen die Eingabe als gültig angesehen haben. Wurde die Eingabe als ungültig anerkannt, so wird auch bei der serverseitigen Programmierung eine Fehlermeldung auf der Webseite angezeigt. Nachdem der Kunde erfolgreich in der Kundendatenbank angelegt wurde, bekommt der Neukunde eine Bestätigungsmail zu gesendet. Der Neukunde kann sich erst erfolgreich am System anmelden, wenn dieser den in der Email enthaltenden Link ausgeführt hat. 

Das Passwort des Kunden wird nicht in Klartext in die Datenbank gespeichert. Bevor es in die Datenbank geschrieben wird, wird das Passwort mit einen \glqq MD5-Hash-Algorithmus\grqq{} unkenntlich gemacht. Der MD5-Hashwert ist nicht mit einen Schlüssel wieder invertierbar. Er kann zur Speicherung von Passwörtern verwendet werden, da beim \glqq MD5-Hash-Algorithmus\grqq{} der gleiche Input immer den gleichen Output erzeugt. Somit wird beim Login-Check einfach auch das eingegebene Passwort unkenntlich gemacht und verglichen, ob es mit den Wert in der Datenbank übereinstimmt.\\ Die \textit{Abbildung 30} zeigt den Registrierungsvorgang nochmal als BPMN dargestellt.\\

\begin{figure}[H]
	\begin{center}
			\includegraphics[width=225pt]{Bilder/Abbildung13_Reg_Vorgang.png}
	\end{center}
	\caption{Der Registrierungsvorgang}
	\label{fig:Abbildung 13}
\end{figure}


\textbf{Der Anmeldungsvorgang}\\
Der Anmeldungsvorgang läuft ähnlich wie der Registrierungsvorgang ab. Der Kunde tippt als Benutzernamen seine Email-Adresse und als Passwort das bei der Registrierung angegebene Passwort in das Anmeldeformular ein. Wie bei der Registrierung überprüft ein JavaScript die Eingabe bevor diese an den Server weitergeleitet wird. Die vorherige Überprüfung ist wieder nur bei aktivierten JavaScript im Webbrowser möglich. Deswegen wird die Überprüfung bei der Anmeldung auch nochmal vom Server überprüft. Hat der Kunde eine gültige Eingabe getätigt, so wird die Email-Adresse und das Passwort mit den Datensätzen in der Kunden-Tabelle der Datenbank verglichen. Um den Vergleich ausführen zu können wird bei der Anmeldung auch das Passwort durch den  \glqq MD5-Hash-Algorithmus\grqq{} unkenntlich gemacht. Wurde in der Tabelle ein Datensatz gefunden, wo Email-Adresse und Passwort übereinstimmen, so wird der Kunde eingeloggt. Wurde keine Übereinstimmung gefunden, so wird auf der Webseite eine Fehlermeldung ausgegeben.\\ 
Der Einlogvorgang wurde wie folgt umgesetzt. Die Anmelde-Email-Adresse wird in einer SESSION-Variable gespeichert. Der SESSION-Variable wurde der Name \glqq \$\_SESSION[ 'angemeldet' ]\grqq{} gegeben. An Hand dieser Variable erkennt der Webshop, ob der Kunde angemeldet ist oder nicht. Wurde die Variable angelegt und enthält einen Wert, so wird dieses als angemeldet interpretiert. Die Abmeldung erfolgt einfach durch das Löschen dieser Variable. Eine SESSION-Variable behält die enthaltenden Daten über mehrere Seitenaufrufe. Wenn nicht mehr auf die Variable zugegriffen wird, werden die Daten dieser Variable automatisch nach einer bestimmten Zeit gelöscht. Ob der Kunde angemeldet ist, muss in einer SESSION-Variable zwischengespeichert werden, da das HTTP-Protokoll verbindungslos ist. Es wird also jede HTTP-Anfrage isoliert betrachtet. Ohne die SESSION-Variable könnte sich somit der Webserver nicht merken, dass der Kunde angemeldet ist. Einer Session wird in PHP eine eindeutige \glqq SESSION-ID\grqq{} zugeordnet. Diese SESSION-ID wird entweder als Cookie im Browser abgespeichert oder immer der Seiten-URL mit übergeben. Wird die SESSION-ID in einem Cookie im Browser abgespeichert, so wird dieser beim Schließen des Browsers gelöscht. Die Daten, die in der SESSION-Variable abgespeichert sind, werden auf der Serverseite zwischengespeichert und nicht im Browser. Die SESSION-ID wird von Browser bei jeder Anfrage an den Server weitergeleitet. An Hand dieser SESSION-ID kann der Server die SESSION-Variable zu den entsprechenden Browser zuordnen.\\

\textbf{Die clientseitige Überprüfung}\\
Wenn JavaScript aktiviert ist, werden, wie oben schon erwähnt, die Formulareingaben zuvor im Browser überprüft. Dafür wird beim Formular den Attribut \glqq onsubmit\grqq{} eine JavaScript-Funktion übergeben. Die Formulareingabe wird hierbei nur an den Server weitergeleitet, wenn diese Methode als Rückgabewert \glqq true\grqq{} zurückgibt. Die JavaScript-Funktion überprüft bei der Eingabe, ob überhaupt ein Text in die Textfelder eingegeben wurde und die Eingabe gültig ist. Zum Beispiel darf eine Postleitzahl nur aus 5 Ziffern bestehen. Die clientseitige Programmierung findet bei der Anmeldung in der Datei \glqq /Funktions/JS/anmeldung.js\grqq{} und bei der Registierung in der Datei \glqq /Funktions/JS/registrierung.js\grqq{} statt.

\newpage
\begin{center}
	\begin{lstinputlisting}[language=HTML, caption={Login-Formular (vereinfacht)}]
		{Quellcode/login-formular.php.inc}
	\end{lstinputlisting}
\end{center}

Der \textit{Quellcode 2} zeigt in vereinfachter Version das Login-Formular. Beim \glqq form\grqq{}-Tag kann das \glqq onsubmit\grqq{}-Attribut gefunden werden. Dieses Attribut startet beim Klick auf den Submit-Button die enthaltende JavaScript-Funktion. Das Attribut lässt nur die Durchführung des Submits zu, wenn die JavaScript-Funktion als Rückgabewert den Wert \glqq true\grqq{} zurückliefert.

Um unnötigen Quelltext zu umgehen, wurde ein Teil des Eingabeüberprüfungsquellcode in eine andere JavaScript-Datei ausgelagert. Der ausgelagerte Quellcode wird bei jeden Formular benötigt. Dieser Quellcode überprüft, ob alle Pflichtfelder ausgefüllt sind. Die ausgelagerte Funktion trägt den Namen \glqq sindAlleFelderAusgefuellt\grqq{} und befindet sich in der Datei \glqq /Funktions/JS/vorcheck\_std\_funktionen.js\grqq{}. Der Funktion werden beim Aufruf drei Parameter übergeben. Der erste Parameter erhält ein Array. Dieses Array enthält die Feldnamen der Textfelder, wovon die Eingabe überprüft werden soll. Der zweite Parameter enthält wiederum ein Array. Dieses Array enthält die Meldungswörter, die in der Fehlermeldung für das jeweilige Textfeld eingesetzt werden sollen. Die Meldungswörter werden in die Fehlermeldung eingesetzt, wenn das Textfeld nicht gefüllt ist. Beim letzten Parameter wird der Name des Formulars angeben, worauf sich die Textfelder befinden. Sind nicht alle Felder ausgefüllt, erstellt die Funktion eine Fehlermeldung und gibt diese als Rückgabewert aus.\\
In die Auslagerungsdatei wurden zudem Funktionen zur Überprüfung der Email-Adresse oder ob die Eingabe eine Zahl ist abgelegt. Die Funktionen in den Dateien \glqq /Funktions/JS/anmeldung.js\grqq{} und  
\glqq /Funktions/JS/registrierung.js\grqq{} rufen die ausgegliederten Funktionen auf und führen noch eine genauere Überprüfung durch. Die Funktionen testen zum Beispiel, ob das Passwort mit der Passwortwiederholung übereinstimmt.\\

\textbf{Serverseitige Programmierung}\\
Wie bei der clientseitigen Programmierung wurde auch bei der serverseitigen Programmierung die Methode zur Überprüfung ob alle Pflichtfelder ausgefüllt sind ausgelagert. Bei der serverseitigen Programmierung wurde hierfür die Technik der Vererbung genutzt. Serverseitig wurde der Webshop objektorientiert programmiert. Für die zuvor genannte Methode wurde die Klasse \glqq EingabeCheckGrundlegend\grqq{} erstellt. Diese Klasse wurde an alle anderen Klassen vererbt, die Formulareingaben entgegen nehmen müssen. Für die Registrierung, Anmeldung und Änderung des Kundenprofils ist die Klasse \glqq Kunde\grqq{} entwickelt worden. Sie enthält Methoden, mit der die Formulareingabe überprüft, ein neuer Kunde angelegt, ein Kunde verwaltet und die Anmeldung durchgeführt werden kann.\\
Instanziiert wird die Klasse durch Skripte des Ordners \glqq /Funktions/PHP\grqq{}. Registriert sich ein neuer Kunde am Webshop, so wird von den Registierungsformular der Skript \glqq registrierung\_durchfuehren.php.inc\grqq{} des zuvor genannten Ordners aufgerufen. Dieser Skript instanziiert die Klasse \glqq Kunde\grqq{}, überprüft die Formaulareingabe und legt bei gültiger Eingabe den Kunden an. Zur Überprüfung der Eingabe und zum Anlegen des Kunden ruft der Skript die benötigten Methoden der Klasse \glqq Kunde\grqq{} auf.\\
Weitere Skripte, die mit der Klasse \glqq Kunde\grqq{} arbeiten, sind zum Beispiel die Skripte \glqq anmeldung\_durchfuehren.php.inc\grqq{}, \glqq kunde\_aktivieren.php.inc\grqq{} und \glqq kunde\_aendern.php.inc\grqq{}. Der erste der Skript der Liste ist für die Anmeldung des Kunden zuständig, der zweite aktiviert den Kunden, wenn der Aktivierungslink der Aktivierungsemail angeklickt wurde und der letzte Skript ist für das Ändern der Anschrift oder Benutzerkennung des Kunden zuständig.\\
Die Klasse \glqq Kunde\grqq{} legt alle Daten in der Datenbank ab. Wird z.B. die Instanzmethode \glqq get\_vorname\_from\_kid(\$kid)\grqq{} aufgerufen, so fragt die Methode über einen Datenbank-Select den Vornamen des Kunden von der Datenbank ab. Das Java-Schnittstellenprogramm der SAP-Gruppe lauscht die ganze Zeit, ob sich etwas an der Datenbank ändert. Kam es zu einer Änderung in der Datenbank, so wird die Änderung im SAP-System eingetragen.\\

\textbf{Der Loginbereich im Header}\\
Der Loginbereich im Header wurde mit einen sogenannten \glqq Flyout\grqq{} umgesetzt (siehe \textit{Abbildung 31}). Ein Flyout blendet ein Fenster auf der Webseite ein, wenn über ein bestimmtes Symbol mit der Maus gefahren wird. Der Loginbereich bei der Webseite wird also erst angezeigt, wenn sich die Maus auf den Anmeldesymbol befindet. 

\begin{figure}[H]
	\begin{center}
			\includegraphics[width=75mm]{Bilder/Abbildung14_Loginbereich.png}
	\end{center}
	\caption{Flyout: Loginbereich}
\end{figure}

Die Besonderheit bei den Flyout des Webshops ist, dass es auch bei deaktivierten JavaScript funktioniert.  Das Flyout wurde mit den in CSS 2.0 eingeführten \glqq Kindselektor\grqq{} umgesetzt. Mit diesen \glqq Kindselektor\grqq{} ist es eingeschränlt möglich mit CSS eventbasiert zu programmieren. Um diese CSS-Funktion nutzen zu können, muss aber folgende Bedingung erfüllt sein: Das Fenster, dass angezeigt werden soll muss ein Kindelement des Elements sein, welches das Ereignis auslöst. Die Syntax des \glqq Kindselektors\grqq{} sieht wie im \textit{Quellcode 3} zeigt aus. Am Anfang steht die Bedingung, die am Elternelement erfüllt sein muss. Beim Webshop wäre es die Bedingung, dass die Maus sich auf den Anmeldesymbol befinden soll. Diese Bedingung sieht in CSS geschrieben folgendermaßen aus: \glqq .accountCell:hover\grqq{}. Darauf folgt ein \glqq >\grqq{}-Zeichen. Dieses Zeichen legt fest, dass bei erfüllter Bedingung nicht die Style-Anweisung des Elternelements verändert werden soll, sondern die Style-Anweisung eines Kindelementes. Nachfolgend wird das Kindelement angegeben, wo die Style-Anweisung geändert werden soll. Der letzte Teil der Style-Anweisung legt das neue Aussehen des Kindelements fest. Beim Webshop wird hier festgelegt, dass das Kindelement sichtbar werden soll. Ein Paar Zeilen vorher definiert eine Anweisung, dass das Kindelement unsichtbar sein soll. Bei gültiger Bedingung wird diese Anweisung einfach von der eben genannten Anweisung überschrieben.

\begin{center}
	\begin{lstinputlisting}[language=CSS, caption={Loginbereich: Umsetzung des Flyouts mit CSS 2.0}]
		{Quellcode/flylayout.css}
	\end{lstinputlisting}
\end{center}

\textbf{Kundenprofil ändern}\\
Hat sich eine Kunde angemeldet, so bietet der Webshop ihn die Möglichkeit die bestehenden Login-Daten und die persönliche Anschrift zu ändern. Erreicht können die Formulare zur Änderung der Daten über ein Flyout. Wenn der Kunde angemeldet ist wird das Flyout unter den Anmeldesymbol zu einem Menü geändert (siehe \textit{Abbildung 32}).
\begin{figure}[H]
	\begin{center}
			\includegraphics[width=55mm]{Bilder/Abbildung15_Menue_Profil_aendern.png}
	\end{center}
	\caption{Flyout: Kunde angemeldet}
\end{figure}
Bei diesem Menü muss der Eintrag \glqq Profil verwalten\grqq{} gewählt werden, um die bestehenden Login-Daten und die persönliche Anschrift ändern zu können. Nach der Anwahl des Eintrags bekommt der Kunde die in der \textit{Abbildung 33} gezeigte Seite zu sehen.
\begin{figure}[H]
	\begin{center}
			\includegraphics[width=130mm]{Bilder/formulare_profil_aendern.png}
	\end{center}
	\caption{Formulare Kundenprofil ändern}
\end{figure}
Die Änderung des Kundenprofils wurde mit Hilfe von zwei Formularen umgesetzt. Das erste Formular ändert die Anschrift des Kunden und das andere die Logindaten. Der Vorteil dieser zwei Formulare ist, dass somit die Seite übersichtlicher gestaltet ist und meist der Kunde  nur die Adresse ändert, wenn er umgezogen ist. Häufiger wird es wahrscheinlich vorkommen, dass der Kunde das Loginpasswort ändert. Durch die Trennung der Adresse und der Logindaten werden bei Änderung des Passworts nur das neue Passwort und die Login-Email-Adresse an den Server übertragen und neu in die Datenbank geschrieben. Somit werden so wenig wie möglich unnötige Daten zum Server übertragen.\\
Technisch umgesetzt wurde der Vorgang wie beim Registrierungsvorgang. Wenn JavaScript aktiviert ist, überprüft die JavaScript-Funktion \glqq vorcheckEingabeEmailPasswort()\grqq{} oder \glqq vorcheckEingabeAdresse()\grqq{} browserseitig die Eingabe. Die aufgerufene Funktion ist abhängig von welchen Formular der Submit-Button geklickt wurde. Diese Funktionen sind in der Datei \glqq /Funktions/JS/ \\ vorcheck\_profil\_aendern.js\grqq{} vorzufinden. War die Überprüfung nicht erfolgreich, wird wieder eine Fehlermeldung auf den entsprechenden Formular ausgegeben. War die Überprüfung erfolgreich wird aus den bekannten Gründen diese nochmal auf dem Server durchgeführt. War die serverseitige Überprüfung erfolgreich, ruft die Funktionsdatei \glqq /Funktions/PHP/kunde\_aendern.php.inc\grqq{} die entsprechende Änderungsmethode der Klasse \glqq Kunde\grqq{} auf und übergibt der Methode als Parameter die Formulareingabe. Diese Methode ändert die entsprechenden Werte in der Datenbank mit Hilfe einer Update-SQL-Anweisung. Wurde die Änderung erfolgreich ausgeführt, wird der Kunde über eine Informationsbox darüber informiert.

\subsubsection{Arbeitspaket 7: Entwickeln einer Suchfunktion}
Dieser Abschnitt beschreibt, wie die Suche beim Projekt umgesetzt wurde. Der Webshop hat eine \glqq normale\grqq{} und eine \glqq erweiterte\grqq{} Suche erhalten. Zudem wurde eine abstrakte Klasse entworfen, die Methoden für die Auflistung der Suchergebnisse bietet. Dieses Klasse wurde an die Klasse für die Suche vererbt und konnte ebenso von Herrn Schnürer für die Auflistung der über die Navigation gefundenen Produkte einer Kategorie verwendet werden.\\

\textbf{Suchwortvervollständigung}\\
Die Suchfunktion wurde beim Webshop mit einer Suchwortvervollständigung ausgerüstet. Diese Vervollständigungsfunktion ist mit Ajax programmiert worden. Mit Ajax kann der Browser über JavaScript, während eine Webseite geöffnet ist, Anfragen an den Server senden und die Antwort des Servers entgegen nehmen. Für den Webserver sind diese Anfragen wie ganz normale Seitenaufrufe. Will der Browser Parameter zur Anfrage hinzufügen werden dieses über \glqq GET\grqq{} übermittelt. Dieses heißt, dass die Parameter an die Adressenzeile angehängt werden.\\
Programmiert wurde die Suchwortvervollständigung, wie es in den folgenden Zeilen beschrieben wird. Beim Textfeld ruft das Attribut \glqq onkeyup\grqq{} eine JavaScript-Funktion der Datei \glqq /Funktions/JS/suche.js\grqq{} auf. Diese Funktion sendet mit AJAX eine Anfrage an den Server. Bei dieser Anfrage fragt der Browser den Server an, ob zu der Eingabe etwas in der Datenbank gefunden werden kann. Mit der Zeile \glqq html2.onreadystatechange = autovervollstaendigung\_normale\_suche\_result\_erweiterte\_suche;\grqq{} wird die Funktion festgelegt, die der JavaScript aufrufen soll, falls der Server antwortet. Diese Funktion nimmt den Rückgabewert des Servers entgegen und gibt diesen in einen DIV auf der Webseite aus.  Der Rückgabewert ist mit Hyperlinks versehen. Dieses Hyperlinks sorgen dafür, dass bei Anwahl eines der Rückgabewerte das Suchtextfeld mit diesen Wert gefüllt wird und somit das Suchwort vervollständigt wurde. Die serverseitige Umsetzung der Suchwortvervollständigung findet in der Datei \glqq /Funktions/PHP/ajax\_autovervollstaendigung\_suche.php\grqq{} statt. Diese Datei instanziiert eine für die Suche entwickelte Klasse, wenn beim Aufruf der Datei ein Suchwort via GET-Parameter übergeben wurde. Die entwickelte Klasse trägt den Namen  \glqq Suche\grqq{}. Nach der Instanziierung ruft die Datei eine Suchmethode der Klasse auf und gibt das Suchergebnis mit den Befehl \glqq echo\grqq{} aus. Der Suchmethode wird als Parameter der Suchbegriff übergeben. Die aufgerufenen Methode ist abhänig von der Suchfunktion. Wurde die Anfrage von der \glqq normalen\grqq{} Suchfunkton gesendet wird die Methode \glqq normale\_suche\_autovervollstaendigung()\grqq{} aufgerufen. Ist der Absender der Anfrage die \glqq erweiterte\grqq{} Suche, so wird die Methode \glqq erweiterte\_suche\_autovervollstaendigung()\grqq{} aufgerufen. Die Ausgabe der Datei wird später von den JavaScript als den eben genannten Rückgabewert entgegengenommen. \\

\textbf{Die eigentliche Suchfunktion}\\
Beim Webshop wurden zwei Suchfunktionen implementiert. Eine \glqq normale\grqq{} Suche im Header der Webseite und eine \glqq erweiterte\grqq{} Suche. Bei der erweiterten Suche (siehe \textit{Abbildung 34}) kann der Kunde die Suche noch genauer spezifizieren. Der Kunde kann z.B. Angeben machen, um was für eine Bauart von Fahrrad es sich handeln soll (Mountainbike, Rennrad,...). Der Suchvorgang läuft bei beiden Suchfunktionen gleich ab. Wird der Submit-Button geklickt, wird zunächst erst über JavaScript überprüft, ob die Eingabe gültig ist. Wie schon beim letzten Arbeitspaket erläutert dient die browserseitig Überprüfung mit JavaScript nur zur Einsparung von unnötigen Datenverkehr zwischen Server und Browser. Aus den von letzten Arbeitspaket bekannten Grund, wird die Eingabeüberprüfung nochmal mit PHP durchgeführt. War die Eingabeüberprüfung mit PHP erfolgreich, wird die, für die Suchfunktion entsprechende Methode der Klasse \glqq Suche\grqq{} aufgerufen. Die Suche wird innerhalb der Methoden rein in SQL umgesetzt. Bei SQL-Datenbanken ist die Suche praktisch schon fertig erhältlich. Der Programmierer muss nur einen entsprechende Abfrage formulieren und bekommt von der Datenbank die entsprechenden Rückgabewerte. Wird bei SQL der \glqq like\grqq{}-Operator verwendet, so unterstützt SQL auch Platzhalter für mehre Zeichen. Diese wurden bei den Suchfunktionen benutzt, um festzulegen, dass hinter und vor den Suchbegriff noch andere Zeichen stehen dürfen.

\begin{figure}[H]
	\begin{center}
			\includegraphics[width=115mm]{Bilder/erweiterte_suche.png}
	\end{center}
	\caption{Formular: Erweiterte Suche}
\end{figure}

\textbf{Anzeige der Suchergebnisse}\\
Bei der Anzeige der Suchergebnisse wurde versucht den Quelltext so zu programmieren, dass Herr Schnürer so viel, wie möglich für die Navigation übernehmen konnte. Es wurde somit versucht so viel, wie möglich in eine Klasse auszulagern. Dieses Klasse wurde so entwickelt, dass diese an die Navigations- und Suche-Klasse vererbt werden konnte. Die vererbten Methoden mussten also nur einmal programmiert werden. Der Klasse wurde der Namen \glqq Artikelauflistung\grqq{} gegeben. Sie ist in der Datei \glqq /Classes/artikel.php.inc\grqq{} zu finden. Die Klasse enthält unter anderen eine Methode, die als Parameter den direkten Rückgabewert des Datenbankservers auf eine SQL-Anweisung entgegennimmt und aus den Rückgabewert eine übersichtliche Liste der Produkte erstellt. Die Liste zeigt den Produktnamen, die vorhandenen Exemplare, die Produkt-Kategorie, den Preis und die Bewertung des Produktes an. Zudem ist ein Bild des Produktes zu sehen. Damit die Methode eine Leiste zur Seitenwahl am Ende der Webseite anzeigen kann, müssen der Methode noch weitere Parameter übergeben werden. Es wird in einen Parameter übergeben, welche Seitennummer aktuell aufgerufen ist. Ein anderer Parameter nimmt entgegen, wie viele Seiten insgesamt zum Suchergebnis gefunden wurden. \\
Die Methode kann nur den Datenbankrückgabewert verarbeiten, wenn dieser die festgelegten Feldnamen enthält. Da bei der Navigation und der Suche die SQL-Anweisung fast gleich ist, wurde der gleiche Teile in eine Methode ausgelagert. Diese Methode gibt somit den vorderen Teil der SQL-Anweisung als Rückgabe aus. Für die Suche und die Navigation muss in der entsprechenden Klasse also nur noch der Rückgabewert dieser Methode durch den \glqq WHERE-Teil \grqq{} der Anfrage ergänzt werden.\\
Die \textit{Abbildung 35} zeigt wie ein Suchergebnis des Webshop aussieht. Beim gezeigten Suchergebnis wurde nach Fahrrädern im Preisbereich zwischen 250 und 400 € gesucht.

\begin{figure}[H]
	\begin{center}
			\includegraphics[width=130mm]{Bilder/suchergebnis.png}
	\end{center}
	\caption{Suchergebnis}
\end{figure}

\subsubsection{Zum Teil übernommenes Arbeitspaket: Warenkorb}
Auf Grund von mehreren Fehlstunden des Gruppenmitglieds Raphael Gregarek war es beim Projektverlauf zu größeren zeitlichen Rückständen beim Arbeitspaket \glqq Warenkorb\grqq{} gekommen. Da die Arbeitspakete von Benedikt Brüntrup schon weitgehend fertig waren, wurde somit die restlichen Aufgaben des Arbeitspaketes \glqq Warenkorb\grqq{} von Herrn Brüntrup übernommen. Die Einarbeitungszeit in den bestehenden Quellcode war recht kurz, da dieser schon von Hilfestellungsarbeiten bekannt war. In den folgenden Zeilen wird beschrieben welche Funktionen des Arbeitspaketes Herr Brüntrup umgesetzt hat.\\

\textbf{Eingabeüberprüfung}\\
Wie bei den anderen Formularen wurde auch beim Warenkorb eine Eingabeüberprüfung programmiert. Diese wurde clientseitig in der Datei \glqq /Funktions/JS/vorcheck\_warenkorb.js\grqq{} und serverseitig in der Datei \glqq /Funktions/PHP/warenkorb.php.inc\grqq{} umgesetzt. Die Überprüfung wurde, wie bei den anderen Formularen umgesetzt und wird deswegen nicht weiter in dieser Ausarbeitung erläutert. Die Eingabeüberprüfung überprüft, ob überhaupt eine Menge angegeben wurde und ob diese im gültigen Bereich liegt. Es können somit keine negative Mengen und auch nicht mehr Produkte, als auf Lager sind, bestellt werden. Zudem wird als Wert nur eine ganze Zahl unterstützt. Die \textit{Abbildung 36} zeigt die Fehlermeldung der Eingabeüberprüfung, wenn der Kunde mehr Artikel bestellen will, als auf Lager sind.\\

\begin{figure}[H]
	\begin{center}
			\includegraphics[width=55mm]{Bilder/warenkorb_eingabepruefung.png}
	\end{center}
	\caption{Eingabeüberprüfung Warenkorb}
\end{figure}

\textbf{Warenkorbübersichtsseite}\\
Bei der Warenkorbübersichtsseite handelt es sich um eine Seite, wo der Kunde den Warenkorbinhalt genausten aufgelistet bekommt. Auf dieser Seite kann der Kunde die Bestellmenge nachträglich verändern oder das Produkt ganz aus dem Warenkorb entfernen.
Zur Umsetzung dieser Seite wurden die von Herrn Gregarek schon entwickelten Klassen um Methoden erweitert. Herr Gregarek hatte bis zum Zeitpunkt der Erstellung der Warenkorbübersichtsseite schon ein Flyouts im Header der Webseite entwickelt, welches den Kunden übersichtlich die Artikel anzeigt, die sich im Warenkorb befinden. Das Flyout dient nur als Schnellübersicht und bietet keine Funktionen, um den Inhalt des Warenkorbs bearbeiten zu können. Die \textit{Abbildung 37} die Warenkorbübersichtsseite.

\begin{figure}[H]
	\begin{center}
			\includegraphics[width=130mm]{Bilder/warenkorb.png}
	\end{center}
	\caption{Warenkorbübersichtsseite}
\end{figure}

\textbf{Änderung der Bestellmenge}\\
Die Änderung der Bestellmenge erfolgt durch einen Klick auf dem Hyperlink mit dem Stift-Symbole. Um die Webseite nicht ganz so altmodisch wirken zu lassen, aber trotzdem bei abgeschalteten JavaScript lauffähig zu haben, wurde sich entschlossen die Mengenänderung mit einen Dialog-Fenster zu lösen. Der Hyperlink lädt die aktuelle Webseite neu und gibt als \glqq GET-Parameter\grqq{} an, dass die Menge des Artikels mit der Artikelnummer x geändert werden soll. Die Datei \glqq /Funktions/warenkorb.php,inc\grqq{} nimmt die Parameter entgegen und ruft ein für vorherige Projekte entwickeltes Dialog-Fenster auf. Das Dialog-Fenster schwebt durch die Style-Anweisung \glqq position:absolute;\grqq{} über der  Webseite. Beim Klicken des \glqq OK\grqq{}-Buttons wird die Seite wieder neugeladen und über \glqq POST\grqq{} die neue Menge des Artikels an den Server übertragen. Wie dieses Dialog-Fenster aussieht kann der \textit{Abbildung 38} entnommen werden. Bei gültiger Eingabe ruft der Server die Methode \glqq aendere\_bestellmenge\_artikel\_im\_warenkorb(\$pid, \$menge)\grqq{} auf und ändert die Mengenangabe im Warenkorb.

\begin{figure}[H]
	\begin{center}
			\includegraphics[width=70mm]{Bilder/mengenaenderung.png}
	\end{center}
	\caption{Dialog: Bestellmenge ändern}
\end{figure}

\textbf{Löschen eines Artikels aus dem Warenkorb}\\
Um ein Artikel aus dem Warenkorb zu löschen wurden zwei Methoden umgesetzt. Bei der ersten Methode handelt es sich um einen Hyperlink, der sich hinter jeder Artikelzeile auf der Warenkorbübersichtsseite befindet. Der Hyperlink wird auf der Seite durch ein rotes Kreuz dargestellt. Dieser Hyperlink lädt die Seite neu und übergibt als Parameter, dass der Artikel mit der Artikelnummer X gelöscht werden soll. Die Datei \glqq /Funktions/warenkorb.php,inc\grqq{} nimmt wieder den Parameter entgegen und ruft ein Dialog-Fenster auf (siehe \textit{Abbildung 39}). Beim Dialog-Fenster muss der Kunde bestätigen, dass wirklich der Kunde gelöscht werden soll. Der \glqq OK-Button\grqq{} des Dialogs ist mit einen Hyperlink gelöst worden. Dieser Hyperlink lädt die Seite neu und übergibt als Parameter, dass der Artikel wirklich gelöscht werden soll. Die Datei \glqq /Funktions/warenkorb.php,inc\grqq{} erkennt den gesetzten Parameter und ruft die Methode \glqq entferne\_artikel\_aus\_warenkorb(\$pid)\grqq{} der Klasse \glqq Warenkorb\grqq{} auf. Diese Methode löscht den Artikel aus dem Warenkorb. Die Variabel \glqq \$pid\grqq{} enthält bei dieser Methode die Artikelnummer des Artikels, der aus dem Warenkorb gelöscht werden soll.\\
Eine weitere Löschfunktion wurde mit Checkenboxen umgesetzt. Durch die Checkboxen kann der Kunde mehrere Artikel löschen. Der Kunde markiert mit den Checkboxen die Artikel, die gelöscht werden sollen. Durch einen Klick auf den Button \glqq Markierte Artikel löschen\grqq{} wird die Markierung an den Server weitergeleitet. Die Weiterleitung findet über die Methode \glqq POST\grqq{} statt. Bei den Button \glqq Markierte Artikel löschen\grqq{} handelt es sich um einen Submit-Button. Die Checkboxen befinden sich in einen HTML-Formular. Dieses Formular überträgt bei einer Bestätigung des Submit-Buttons die Namen  der markierten Checkboxen und die zugeordneten Artikelnummern an das PHP-Skript \glqq /Funktions/warenkorb.php,inc\grqq{}. Das Skript zeigt nach dem Erhalt der Daten auch ein Dialog mit einer Sicherheitsfrage an. Wird diese Frage bestätigt sorgt der Bestätigungs-Hyperlink dafür, dass die Methode \glqq mehrere\_artikel\_aus\_warenkorb\_entfernen(\$arr\_pids)\grqq{} der Klasse \glqq Warenkorb\grqq{} aufgerufen wird. Diese Methode löscht alle als Array übergebenen Artikel aus den Warenkorb. Der Parameter \glqq \$arr\_pids\grqq{} enthält somit ein Array mit Artikelnummern der Artikel, die aus dem Warenkorb entfernt werden sollen. Das Array wird aus dem vom Formular übertragenen Daten erstellt.

\begin{figure}[H]
	\begin{center}
			\includegraphics[width=70mm]{Bilder/sicherheitsfrage_artikel_loeschen.png}
	\end{center}
	\caption{Dialog: Sicherheitsfrage Artikel löschen}
\end{figure}

\subsubsection{Zum Teil übernommenes Arbeitspaket: Bestellvorgang}
Da auch der Bestellvorgang etwas hinter den Zeitplan lag, wurde hier auch ein Teil von Herrn Brüntrup übernommen. Es wurde eine Routine programmiert, die nach dem Klicken des Buttons \glqq Zur Kasse\grqq{} abgelaufen wird. In dieser Routine stellt der Kunde alle Einstellungen, die mit der Bestellung zu tun haben, ein. Hierzu gehört die Versandart, eine nochmalige Möglichkeit den Warenkorb zu ändern, die Zahlungsart und die Lieferadresse. Die letzte Seite der Bestellroutine zeigt nochmal alle Einstellungen übersichtlich auf einer Seite zusammen gefasst. Auf dieser Seite können alle Einstellungen nachträglich verändert werden. Die Übersichtseite entspricht somit den Gesetz des bürgerlichen Gesetzbuches §312g Absatz 1. Bei diesen Gesetz muss den Kunden die Möglichkeit gegeben werden nochmal vor dem Absenden der Bestellung alle Einstellungen bezüglich der Bestellung ändern zu können. Bei online Shops existiert auch eine \glqq Ausgestaltungspflicht des Bestellbuttons\grqq{}. Der Button zum Bestellabschluss muss aus den anderen Button hervorstechen und eindeutig beschriftet sein. Es muss also den Kunden klar werden, dass die Bestellung mit den Button abgeschlossen wird und diese kostenpflichtig ist. Bei diesem Webshop wurde der Button deswegen in einer anderen Farbe gefärbt und mit \glqq Jetzt kostenpflichtig bestellen\grqq{} beschriftet. Der Kunde kann allerdings nur die Bestellung abschließen, wenn dieser die AGB des Webshop akzeptiert hat. Somit wird die AGB ebenso auf der Bestellübersicht angezeigt. Akzeptieren kann der Kunde die AGB indem dieser bei der Checkbox \glqq Ich akzeptiere die AGB.\grqq{} einen Hacken setzt. Wie auf der \textit{Abbildung 40} zu sehen ist wurde bei der Übersichtsseite die Änderungsfunktionen über einen Hyperlink mit Stiftsymbol umgesetzt. Dieser Hyperlink leitet den Kunden zu der Seite der Bestellroutine zurück, wo die entsprechende Einstellung getätigt wurde.

\begin{figure}[H]
	\begin{center}
			\includegraphics[width=100mm]{Bilder/bestelluebersicht.png}
	\end{center}
	\caption{Bestellroutine: Bestellübersicht}
\end{figure}

Beim Klicken auf dem  Bestellbuttons wird die Seite neugeladen. Die neugeladene Seite ruft nun die Methode \glqq zwischengespeicherter\_bestellungslisteneintrag\_bestellen()\grqq{} auf, welche der Klasse \glqq Bestellungen\grqq{} angehört. Diese Methode schließt den Bestellvorgang ab und speichert alle Einstellungen auf der Datenbank ab. Da es gesetzlich vorgeschrieben ist, dass unverzüglich nach dem Absenden der Bestellung diese bestätigt werden muss, versendet der Webshop somit sofort eine Bestätigungs-Email. Zum Absenden der Bestätigungs-Email wurde ebenfalls eine Methode in der Klasse \glqq Bestellungen\grqq{} angelegt. Der Name dieser Methode lautet \glqq sende\_bestaetigungs\_email()\grqq{}. Als Content-Type der Email wurde HTML festgelegt. HTML bietet den Vorteil, dass die Email in den Design des Webshops gestaltet werden konnte und auch Tabellen, Farben und Bilder anzeigen kann. Über die eigentlichen HTML-Ansicht der Email wurde die Style-Anweisung der Email gepackt. Somit muss diese nicht erst vom Webserver geladen werden, wenn der Kunde die Email aufruft. Für das Versenden der Email ist eine andere Klasse zuständig. Der Name dieser Klasse lautet \glqq Email\grqq{}. In dieser Klasse wurde eine Methode implementiert, die sofort schon den richtigen Header für eine Email mit den Content-Type \glqq HTML\grqq{} schreibt und diese über die, in der \glqq Config-Datei\grqq{} festgelegte Email-Adresse versendet. Der Name dieser Methode lautet \glqq html\_email\_senden(\$empfaenger\_adresse, \$betreff, \$html\_code)\grqq{}. Die \textit{Abbildung 41} zeigt einen Ausschnitt der Bestätigungs-Email.

\begin{figure}[H]
	\begin{center}
			\includegraphics[width=125mm]{Bilder/email.png}
	\end{center}
	\caption{Bestätigungs-Email}
\end{figure}


Mit der Übersichtseite inbegriffen besteht die Bestellroutine aus fünf Seiten. Damit die Webseite möglich übersichtlich gestaltet ist, wird deswegen auf jeder Seite nur eine Einstellung vorgenommen. Bei der ersten Seite wird z.B. nur die Versandart festgelegt. Sind für die Einstellung Eingabefenster nötig, so werden diese, wie schon mehrfach erklärt client- und serverseitig überprüft. Es wurde sich dazu entschlossen immer nur eine Einstellung auf eine Seite zu packen, da bei bestehenden Anbietern, wie z.B. Amazon das \glqq Vollpacken\grqq{} der Seiten der Bestellroutine als sehr unübersichtlich angesehen wurde. 
