\newpage
\section{Sicherheit}
Zum Schutz des Webshop vor Angriffen wurden mehrere Sicherheitsmechanismen in die Seite eingebaut. Dieser Abschnitt erläutert welche Sicherheitsmechanismen verbaut wurden und warum diese verbaut worden sind.

\subsection{\glqq .htaccess\grqq{}-Datei}
Als eine mögliche Schwachstelle sah das Projektteam an, dass PHP-Dateien direkt über die Adressenleiste im Browser aufgerufen werden können. Es könnte zum Beispiel somit ein Benutzer durch direktes aufrufen einer Datei, versuchen den Passwortschutz der Webseite zu umgehen. Dieser Benutzer wäre dann vielleicht in der Lage Produkte zu kaufen ohne registriert zu sein. Diese Schwachstelle wurde gelöst indem nur die Index-Datei direkt aufgerufen werden kann. Die Index-Datei erhielt somit als einzige Datei die Endung \glqq *.php\grqq{}. Alle anderen PHP-Skripte bekamen die Dateiendung \glqq *.php.inc\grqq{}. In einer sogenannten \glqq .htaccess\grqq{}-Datei wurde danach festgelegt, dass von Browser aus die \glqq *.php.inc\grqq{}-Dateien nicht geöffnet werden können. Um die Skripte mit \glqq *.php.inc\grqq{}-Endung ausführen zu können, werden diese zum vorgesehenen Zeitpunkt in die Index-Datei inkludiert. Die ganze Webseite wird somit komplett durch die Index-Datei abarbeitet.\\
Des Weiteren wurde mit der \glqq .htaccess\grqq{}-Datei festgelegt, dass der Browser nicht alle Dateien eines Verzeichnisses auflisten darf. Hiermit sollte sichergestellt werden, dass der unerwünschte Benutzer nicht die Ordnerstruktur der Webseite ermitteln kann.

\subsection{Schutz vor: Cross-Site-Scripting (XSS) und SQL Injection}
Als eine weitere Schwachstelle wurde Cross-Site-Scripting (XSS) und SQL Injection angesehen. Bei SQL-Injection versucht eine Angreifer Schadcode in die Datenbank einzuspielen. Hierzu nutzt der Angreifer den Aufbau einer SQL-Anweisung aus. Bei einer SQL-Anweisung werden die zu übergebenen Werte zwischen zwei Hochkomma geschrieben. Der Angreifer schreibt dann in das Textfeld, wovon die Werte in die Datenbank übernommen werden sollen, ein Hochkomma und ein Semikolon. Dadurch geht die SQL-Datenbank davon aus, dass die Datenbankanweisung schon an der Stelle zu Ende ist. In SQL können durch Semikolons mehrere Anweisungen hintereinander geschrieben werden. Somit wäre es möglich, dass der Angreifer nun zum Beispiel eine Anweisung in das Textfeld schreibt, dass die Datenbank gelöscht werden soll und alle Daten weg sind. Zum Schutz vor dieser Situation bietet das \glqq MYSQLi\grqq{}-API die Methode \glqq real\_escape\_string()\grqq{}. Die gerade genannte Methode \glqq escaped\grqq{} alle Hochkomma, sodass diese nicht mehr der SQL-Anweisung schaden können. Cross-Site-Scripting beschreibt einen ähnlichen Vorfall. Hier tippt der Angreifer einen JavaScript in die Textfelder ein. Dieses mal ist von den Angriff der Browser betroffen. Soll nun zu einen späteren Zeitpunkt die Texteingabe auf der Webseite ausgegeben werden, so wird im Browser der JavaScript ausgeführt. Zum Schutz vor einer solchen Situation bietet PHP die Funktion \glqq htmlspecialchars\grqq{}. Mit der Funktion \glqq htmlspecialchars\grqq{} werden Steuerzeichen von HTML ( \&, ", <, >) durch die HTML-Ersatzschreibweise ersetzt. Die Ersatzschreibweise hat zu folge, dass der JavaScript nicht mehr ausgeführt wird. Der \glqq Schadcode\grqq{} wird nun einfach als Text auf der Webseite ausgegeben.